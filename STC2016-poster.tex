\documentclass[25pt, portrait]{tikzposter}
\usepackage{authblk}
\usepackage{tikz}
\usepackage{adjustbox}
\usepackage{textcomp}
\usepackage{multicol}
\usetikzlibrary{shapes, arrows.meta}
\usetheme{Default}

\title{\parbox{.98\textwidth}{\centering Femtosecond-laser induced dynamics of CO on Ru(0001):
New insights from a hot-electron, electronic friction model including surface motion\vspace{.85cm}}}
\author[1,2]{\underline{Robert Scholz}}
\author[1]{Gereon Flo\ss{}}
\author[1]{Peter Saalfrank}
\author[3]{Gernot F\"uchsel}
\author[4]{Ivor Lon\v{c}ari\'c}
\author[4,5,6]{J. I.~Juaristi}
\affil[1]{Institut f\"ur Chemie, Universit\"at Potsdam, Karl-Liebknecht-Str. 24-25, D-14476 Potsdam, Germany}
\affil[2]{Fritz-Haber-Institut der Max-Planck-Gesellschaft, Faradayweg 4-6, D-14195 Berlin, Germany}
\affil[3]{Universiteit Leiden, Gorlaeus Laboratories, Einsteinweg 55, 2333 Leiden, The Netherlands}
\affil[4]{Centro de F\'{\i}sica de Materiales CFM/MPC (CSIC-UPV/EHU), Paseo Manuel de Lardizabal 5, 20018 Donostia-San Sebasti\'an, Spain}
\affil[5]{Departamento de F\'{\i}sica de Materiales,
Facultad de Qu\'{\i}micas, 
Universidad del Pa\'{i}s Vasco (UPV/EHU), Apartado 1072, 20080 San Sebasti\'an, Spain}
\affil[6]{Donostia International Physics Center DIPC, P. Manuel de Lardizabal 4, 20018 San Sebasti\'an, Spain}

\makeatletter
\def\maketitle{\AB@maketitle}
\makeatother
\renewcommand{\Authfont}{\LARGE}
\renewcommand{\Affilfont}{\normalsize}

\settitle{ \centering \vbox{
\@titlegraphic \\[\TP@titlegraphictotitledistance] \centering
\color{titlefgcolor} {\bfseries \huge \sc \@title \par}
\vspace*{1em}
{ \@author \par}
}}


\begin{document}
  \maketitle[width=0.99\textwidth]
  \begin{columns}
    \column{0.42}
      \block{Introduction}{
		  \innerblock{Motivation}{
			 \begin{itemize}
				\item research on small molecules adsorbed to metals is important for:
				\begin{itemize}
				  \item catalytic applications 
				  \item fundamental understanding of bonding
				\end{itemize}
				\item femtosecond(fs)-lasers are a valuable tool for such research as they
				\begin{itemize}
				  \item allow for investigations on small timescales   
				  \item open up new processes compared to heating (femtophotochemistry)
				  \item may enable specific control over catalytic reactions (photocatalysis) 
				\end{itemize}
			 \end{itemize}
		  }

~
		  \innerblock{How does fs-laser-irradiation affect metal surfaces?}{
% 		  \innerblock{How fs-laser-irradiation affects metal surfaces}{
% 		  \innerblock{What happens upon FL-irradiation of metal surfaces?}{
			 \includegraphics[width=.90\colwidth]{abb/modifiedSurfScheme.png}		    
		    \begin{minipage}[t]{.62\colwidth}
				\begin{itemize}
				  \item metals: 
			    %can be described as 
% 			    consist of an
					 ion lattice plus quasi-free electron gas 
				  \item visible light is absorbed only by the electrons %only
% 				  \item Fermi-Dirac schon hier?
						  % \vbox syntax? (for the graphic)
				  \item electrons transfer part of energy to ion lattice, via \includegraphics[height=.8\baselineskip]{abb/one.png}
				  \hspace{-0.4cm} \textbf{\color{innerblocktitlebgcolor!80!orange}electron-phonon coupling}\newline (phonons = lattice vibrations)
				  \begin{itemize}
					 \item electrons couple to phonons as their fast movement causes ``shockwaves'' in ion lattice
					 \item equilibration process completes after $\sim$1$\,$ps			
				  \end{itemize}
				  \item Thus, with fs-lasers, two temperatures emerge:
				  \begin{itemize}
					 \item $T_\mathrm{el}$ - electron temperature
					 \item $T_\mathrm{ph}$ - phonon temperature
% 					 \item Two-Temperature Model (TTM)\cite{anisimov:74}
				  \end{itemize}
% 				  \item can be described with the Two-Temperature Model (TTM)
				\end{itemize}
		    \end{minipage}
		    \begin{adjustbox}{valign=t}
		      \begin{minipage}[t]{.28\colwidth}
				  \begin{flushright}
					 \includegraphics[width=.18\colwidth]{abb/elgas.png}$\,$\\[.5\baselineskip] 
					 \includegraphics[width=.26\colwidth]{abb/TTM_1pulse.eps}
				  \end{flushright}
		      \end{minipage}
		    \end{adjustbox}
			 \begin{itemize}
				\item time evolution can be simulated with a Two-Temperature Model %(TTM)
 \cite{anisimov:74}%consisting of two differential Equations 
			 \end{itemize}
		  }
% 		  \includegraphics[width=.4\colwidth]{abb/TTM_1pulse.eps}
      }
		\block{Models and Methods}{
		  
		}
  \end{columns}
      %   \bibliography{/perm/scholz/phd/lit/biblio}
  \block{}{
    \begin{multicols}{2}
      \begin{thebibliography}{X}

	\bibitem{anisimov:74}
	S. I. Anisimov, B. L. Kapeliovich, and T. L. Perel'man,
	\emph{Sov. Phys.-JETP} \textbf{39}, 375 (1974).

	\bibitem{nilsson:13}
	 M. Dell'Angela, T. Anniyev, M. Beye et al., 
	 \emph{Science} \textbf{339}, 1302 (2013).	 
	
      \end{thebibliography}  
    \end{multicols}
  }
\end{document}


% \makeatletter
% \renewenvironment{tikzfigure}[1][]{
%   \def \rememberparameter{#1}
%   \vspace{10pt}
%   \refstepcounter{figurecounter}
%   \begin{center}
%   }{
%     \ifx\rememberparameter\@empty
%     \else %nothing
%     \\[10pt]
%     {\large Fig.~\thefigurecounter: \rememberparameter}
%     \fi
%   \end{center}
% }
% \makeatother
